\addcontentsline{toc}{section}{СПИСОК ИСПОЛЬЗОВАННЫХ ИСТОЧНИКОВ}

\begin{thebibliography}{9}
	\bibitem{plant1} Ходж, Д. Ботаника для садоводов / Д. Ходж – Москва~: Азбука-Бизнес, 2024. - 224 с. – ISBN 978-5-389-13415-7. – Текст~: непосредственный.
	
	\bibitem{plant2} Кошкин, Е. И. Патофизиология сельскохозяйственных культур / Е. И. Кошкин – Москва~: РГ-Пресс, 2024. - 304 с. – ISBN 978-5-9988-1405-1. – Текст~: непосредственный.	
	
	\bibitem{plant3} Сергеева, М. Н. Культурные растения / М. Н. Сергеева – Москва~: Проспект, 2023. - 120 с. – ISBN 978-5-392-37362-8. – Текст~: непосредственный.	
	
	\bibitem{plant4} Абросимов, В.К. Интеллектуальные сельскохозяйственные роботы/В.К. Абросимов, А.Н. Райков – Москва~: Карьера Пресс, 2022. - 512 с. – ISBN 978-5-00074-318-8. – Текст~: непосредственный.	

	\bibitem{plant5}  Гриценко, В. В. Меры борьбы с болезнями и вредителями растений в сооружениях защищенного грунта /В. В. Гриценко, И. М. Митюшев, Ю. М. Стройков  – Москва~: Академия, 2020. - 160 с. – ISBN 978-5-44-689434-5. – Текст~: непосредственный.	
	
	\bibitem{plant6} Ерофеева, Т. В. Сельскохозяйственная экология /Ерофеева Т. В., Фадькин Г. Н., Чурилова В. В.  Санкт-Петербург~: Лань, 2025. - 100 с. – ISBN 978-5-507-52249-1. – Текст~: непосредственный.	
	
	\bibitem{plant7} Курбанов, С. А. Сельскохозяйственная мелиорация /С. А. Курбанов  – Санкт-Петербург~: Лань, 2021. - 208 с. – ISBN 978-5-8114-6623-8. – Текст~: непосредственный.	
	
	\bibitem{plant8} Овсинский, И. Е. Новая система земледелия /И. Е. Овсинский  – Москва~: Концептуал, 2023. - 240 с. – ISBN 978-5-907624-99-3. – Текст~: непосредственный.	
	
	\bibitem{plant9} Монторо, Ж. Семена и зерна / Ж. Монторо  – Москва~: Эксмо, 2024. - 256 с. – ISBN 978-5-04-196329-3. – Текст~: непосредственный.
	
	\bibitem{plant10}Копытин, И. П. Ведение сельского хозяйства в Центрально-Нечерноземном округе России / И. П. Копытин  – Санкт-Петербург~: Лань, 2022. - 336 с. – ISBN 978-5-8114-9863-5. – Текст~: непосредственный.	
	
	\bibitem{plant11}Савельев, В. А. Растениеводство / В. А. Савельев  – Санкт-Петербург~: Лань, 2021. - 316 с. – ISBN 978-5-8114-8194-1. – Текст~: непосредственный.	
	
	\bibitem{plant12} Ториков, В. Е. Агрономический контроль в растениеводстве / В. Е. Ториков, О. В. Мельникова/ Санкт-Петербург~:Лань, 2024. - 132 с. - ISBN 978-5-507-49427-9. - Текст: непосредственный.
	
	\bibitem{plant13} Ганичкина, О. А. Секреты огородника. Как получить богатый урожай овощей и зелени на вашем участке / О. А. Ганичкина, А. В. Ганичкин  / Москва~:Эксмо, 2024. - 224 с. - ISBN 978-5-04-211846-3. - Текст: непосредственный.
	
	\bibitem{plant14} Кочелаева, Л. Н. Сеньор Помидор. Выращиваем, ухаживаем и едим / Л. Н. Кочелаева  – Москва~: АСТ, 2024. - 160 с. – ISBN 978-5-17-161963-3. – Текст~: непосредственный.	
	
	\bibitem{plant15} Кузнецова, Е. А. Рассадоводство. Первые шаги к здоровому урожаю / Е. А. Кузнецова – Москва: Издательство АСТ, 2023. - 160 с. – ISBN 978-5-17-157334-8. – Текст: непосредственный.
	
	\bibitem{plant16} Ожехелева, З. Е. Биологические препараты для эффективного садоводства / З. Е. Ожехелева – Москва: ФГБНУ ВНИИСПК, 2022. - 166 с. – ISBN 978-5-6049204-3-5. – Текст: непосредственный.
	
	\bibitem{plant17} Кизима, Г. А. Огород и сад для ленивых. Урожаю быть / Г. А. Кизима – Москва: Издательство АСТ, 2020. - 192 с. – ISBN 978-5-17-120651-2. – Текст: непосредственный.
	
	\bibitem{plant18} Волкова, А. П. Энциклопедия пасленовых. Томат. Перец. Баклажан. Физалис / А. П. Волкова – Москва: Издательство АСТ, 2024. - 416 с. – ISBN 978-5-17-164151-1. – Текст: непосредственный.
	
	\bibitem{plant19} Кизима, Г. А. Рассада для начинающих. Первые шаги к богатому урожаю / Г. А. Кизима – Москва: Издательство АСТ, 2025. - 128 с. – ISBN 978-5-17-164591-5. – Текст: непосредственный.
	
	\bibitem{cv20} Демиденко, А. Введение в Computer Vision: как научить компьютер видеть / А. Демиденко – Москва: Литрес, 2025. – 100 с. – ISBN 978-5-04-720541-0. – Текст: непосредственный.
	
	\bibitem{d21} Рутковская, Д. Нейронные сети, генетические алгоритмы и нечеткие системы / Д. Рутковская, М. Пилиньский, Л. Рутковский – Москва: Горячая линия-Телеком, 2023. – 385 с. – ISBN 978-5-9912-0320-3. – Текст: непосредственный.
	
	\bibitem{d22} Лекун, Я. Как учится машина: Революция в области нейронных сетей и глубокого обучения / Я. Лекун – Москва: Альпина PRO, 2021. – 335 с. – ISBN 978-5-907394-92-6. – Текст: непосредственный.
	
    \bibitem{py23} Мэтиз, Э.Изучаем Python: программирование игр, визуализация данных, веб-приложения. 3-е издание / Э. Мэтиз – Санкт-Петербург: БХВ, 2024. – 352 с. – 978-5-9775-1944-1. – Текст: непосредственный.
    
    \bibitem{cv24} Демиденко, А. ИИ и зрение: как машины понимают изображения / А. Демиденко – Москва: Литрес, 2025. – 80 с. – ISBN 978-5-04-727123-1. – Текст: непосредственный.
    
    \bibitem{py25} Любанович, Б. Простой Python / Б. Любанович – Санкт-Петербург: Питер, 2021. – 592 с. – ISBN 978-5-4461-1701-7. – Текст: непосредственный.
    
    \bibitem{cv24} Амундсен, М. RESTful Web API паттерны и практики / А. Демиденко – Москва: Sprint Book, 2025. – 464 с. – ISBN 978-601-08-4867-2. – Текст: непосредственный.
    
	\bibitem{res1} Определитель болезней томатов: фото, описание, меры борьбы и профилактика : сайт / Огород. – Москва : Огород, 2023 – . – URL: \url{https://www.ogorod.ru/ru/ogorod/tomats/14384/Opredelitel-bolezney-tomatov-foto-opisaniye-mery-borby-i-profilaktika.htm} (дата обращения: 13.05.2025). – Текст: электронный.
	
	\bibitem{res2} Болезни рассады томатов: описание с фото, лечение : сайт / КП. – Москва : КП, 2023 – . – URL: \url{https://www.kp.ru/family/sad-i-ogorod/bolezni-rassady-tomatov/} (дата обращения: 13.05.2025). – Текст: электронный.

	\bibitem{yolo1} Распознавание объектов с помощью YOLO v3 на Tensorflow 2.0 : сайт / Proglib. – Москва : Proglib, 2020 – . – URL: \url{https://proglib.io/p/raspoznavanie-obektov-s-pomoshchyu-yolo-v3-na-tensorflow-2-0-2020-11-08} (дата обращения: 13.04.2025). – Текст: электронный.
	
	\bibitem{yolo2} YOLOv11: улучшения в детекции объектов для реального времени : сайт / Ultralytics, 2024 – . – URL: \url{https://docs.ultralytics.com/ru/models/yolo11/} (дата обращения: 13.05.2025). – Текст: электронный.
	
	\bibitem{yolo3} Детекция объектов с помощью YOLOv5 : сайт / Habr. – Москва : Habr, 2021 – . – URL: \url{https://habr.com/ru/articles/576738/} (дата обращения: 11.05.2025). – Текст: электронный.
	
	\bibitem{yolo4} YOLOv7: пользовательское обнаружение объектов : сайт / Habr. – Москва : Habr, 2022 – . – URL: \url{https://habr.com/ru/articles/700794/} (дата обращения: 11.05.2025). – Текст: электронный.
	
	\bibitem{yolo5} Обнаружение объектов с YOLOv3 на Tensorflow 2.0 : сайт / Habr. – Москва : Habr, 2021 – . – URL: \url{https://habr.com/ru/articles/556404/} (дата обращения: 21.04.2025). – Текст: электронный.
	
	\bibitem{yolo6} Как выполнить обнаружение объектов YOLO с помощью OpenCV и PyTorch в Python : сайт / Waksoft. – Челябинск : Waksoft, 2021 – . – URL: \url{https://waksoft.susu.ru/2021/05/19/kak-vypolnit-obnaruzhenie-obektov-yolo-s-pomoshhyu-opencv-i-pytorch-v-python/} (дата обращения: 11.05.2025). – Текст: электронный.
	
	\bibitem{yolo7} Explore Ultralytics YOLOv11 : сайт / Yolov11, 2024 – . – URL: \url{https://yolov11.com/} (дата обращения: 21.04.2025). – Текст: электронный.
	
	\bibitem{yolo8} Как использовать YOLOv11 для обнаружения объектов : сайт / SO Development. – Москва : SO Development, 2024 – . – URL: \url{https://ru.so-development.org/how-to-use-yolov11-for-object-detection/} (дата обращения: 11.05.2025). – Текст: электронный.
	
	\bibitem{yolo9} YOLO11: A New Iteration of “You Only Look Once”: сайт / viso.ai, 2024 – . – URL: \url{https://viso.ai/computer-vision/yolov11/} (дата обращения: 12.05.2025). – Текст: электронный.
	
	\bibitem{yolo10} YOLOv11: настройка и обучение модели для специфичных задач : сайт / Roboflow, 2024 – . – URL: \url{https://roboflow.com/model/yolo11} (дата обращения: 12.05.2025). – Текст: электронный.

	\bibitem{python1} Документация Python: сайт / Habr. – Москва : Habr, 2024 – . – URL: \url{https://www.python.org/doc/} (дата обращения: 02.03.2025). – Текст: электронный.
	
	\bibitem{python2} Язык программирования Python: особенности и перспективы : сайт / GeekBrain. – Москва : GeekBrain, 2024 – . – URL: \url{https://gb.ru/blog/python/} (дата обращения: 14.05.2025). – Текст: электронный.
	
	\bibitem{python3} Обзор языка программирования Python : сайт / Proglib. – Москва : Code Basics, 2023 – . – URL: \url{https://code-basics.com/ru/blog_posts/obzor-yazyka-programmirovaniya-python} (дата обращения: 11.03.2025). – Текст: электронный.
	
	\bibitem{python4} Библиотеки для веб-разработки на Python : сайт / Code Basics. Skypro – Москва : Skypro, 2023 – . – URL: \url{https://sky.pro/wiki/python/biblioteki-dlya-veb-razrabotki-na-python/} (дата обращения: 11.03.2025). – Текст: электронный.

	\bibitem{flask1} Мега-Учебник Flask : сайт / Habr. – Москва : Habr, 2024 – . – URL: \url{https://habr.com/ru/articles/804245/} (дата обращения: 12.03.2025). – Текст: электронный.
	
	\bibitem{flask2} Проектирование RESTful API с помощью Python и Flask : сайт / Skypro. – Москва : Skypro, 20243 – . – URL: \url{https://sky.pro/media/kak-sozdat-rest-api-na-flask/} (дата обращения: 12.03.2025). – Текст: электронный.
	
	\bibitem{flask3} Документация Flask : сайт / Flask, 2020 – . – URL: \url{https://flask.palletsprojects.com/en/stable/} (дата обращения: 14.05.2025). – Текст: электронный.
	
	\bibitem{cn1} Vatathanavaro, S. White Blood Cell Classification: A Comparison between VGG-16 and ResNet-50 Models / V. ЕSupawit, T. Suchat, P. Kitsuchart  – Bangkok~: РГ- King Mongkut’s Institute of Technology Ladkrabang, 2018. - 2 с. – Текст~: электронный – . – URL: \url{https://site.ieee.org/thailand-cis/files/2018/11/JSCI6-Paper-2.pdf} (дата обращения: 11.05.2025).
		
	\bibitem{flask4} Фреймворк Flask: как он работает и зачем нужен : сайт / Skillbox. – Москва : Skillbox, 2023 – . – URL: \url{https://skillbox.ru/media/code/freymvork-flask-kak-on-rabotaet-i-zachem-nuzhen/} (дата обращения: 14.05.2025). – Текст: электронный.
	
	\bibitem{postgres1} Документация PostgreSQL : сайт / Postgresql, 2025 – . – URL: \url{https://www.postgresql.org/docs/} (дата обращения: 11.05.2025). – Текст: электронный.
	
	\bibitem{postgres2} Обзор баз данных PostgreSQL : сайт / Cnews, 2022 – . – URL: \url{https://market.cnews.ru/news/top/2022-04-25_obzor_baz_dannyh_postgresql} (дата обращения: 11.05.2025). – Текст: электронный.
	
	\bibitem{vkr1} ИИ в сельском хозяйстве : сайт / Sber Developer. – Москва : Sber Developer, 2024 – . – URL: \url{https://developers.sber.ru/help/gigachat-api/ai-agricultural} (дата обращения: 14.05.2025). – Текст: электронный.
	
	\bibitem{vkr2} Цифровые технологии и нейросети в сельском хозяйстве : сайт / It фабрика, 2025 – . – URL: \url{https://it-fabric.ru/catalog/neyroseti/tsifrovye-tekhnologii-i-neyroseti-v-selskom-khozyaystve/} (дата обращения: 12.05.2025). – Текст: электронный.
	
	\bibitem{vkr3} Использование машинного обучения и ИИ в сельском хозяйстве : сайт / БиоТех2030. – Москва : БиоТех2030, 2025 – . – URL: \url{http://biotech2030.ru/kak-ii-i-kompyuternoe-zrenie-mogut-sledit-za-rasteniyami/} (дата обращения: 12.05.2025). – Текст: электронный.
	



\end{thebibliography}
