\section*{ВВЕДЕНИЕ}
\addcontentsline{toc}{section}{ВВЕДЕНИЕ}

Современное сельское хозяйство находится на пороге трансформации. Это обусловлено стремительным развитием цифровых технологий. Повышение продуктивности и качества аграрного производства требует внедрения инновационных решений, способных минимизировать потери и оптимизировать процессы. Одной из ключевых задач остается раннее выявление заболеваний сельскохозяйственных культур, таких как томаты, чья восприимчивость к патологиям существенно влияет на урожайность и экономическую устойчивость хозяйств \cite{plant8}.

Традиционные методы диагностики, основанные на визуальном осмотре или лабораторных исследованиях, часто оказываются трудоемкими и медленными, а также требуют привлечения квалифицированных специалистов. Но уже на сегодняшний день, технологии искусственного интеллекта (ИИ) и компьютерного зрения представляют новые способы автоматизацию процессов мониторинга здоровья растений, которые смогут обеспечить высокую точность и оперативность \cite{sh1}.

Создание удобных инструментов на базе машинного обучения открывает новые возможности в сельском хозяйстве. Такие решения позволят эффективно анализировать визуальные данные, выявлять признаки заболеваний растений и предлагать обоснованные рекомендации по их лечению и профилактике. Интуитивно понятный интерфейс и автоматизация процессов делают данный подход доступным для широкого круга пользователей.

\emph{Цель данной работы} -- разработка интеллектуальной системы для выявления заболеваний сельскохозяйственных растений, в частности томатов, на основе технологий компьютерного зрения и веб-технологий. Для достижения этой цели необходимо решить следующие задачи:

\begin{itemize}
	\item определить требования к системе диагностики и описать её функциональные возможности;
	\item разработать архитектуру программного комплекса;
	\item обучить свёрточную нейронную сеть распознавать заболевания;
	\item реализовать веб-приложение для загрузки изображений растений, анализа заболеваний и предоставления рекомендаций;
	\item провести комплексное тестирование системы для оценки её надежности и точности.
\end{itemize}

\emph{Структура и объем работы.} Отчет состоит из введения, 4 разделов основной части, заключения, списка использованных источников, 2 приложений. Текст выпускной квалификационной работы равен \formbytotal{lastpage}{страниц}{е}{ам}{ам}.

\emph{Во введении} сформулирована цель работы, поставлены задачи разработки, описана структура работы, приведено краткое содержание каждого из разделов.

\emph{В первом разделе} проводится анализ предметной области, включая сбор информации о болезнях томатов и методах их диагностики с использованием искусственного интеллекта.

\emph{Во втором разделе} описываются требования к разрабатываемой системе диагностики заболеваний.

\emph{В третьем разделе} представлены проектные решения для интеллектуальной системы, архитектура модели машинного обучения и веб-приложения.

\emph{В четвертом разделе} приведены программные компоненты системы, их описание и результаты тестирования.

В заключении излагаются основные результаты работы, полученные в ходе разработки.

В приложении А представлен графический материал.

В приложении Б представлены фрагменты исходного кода. 
